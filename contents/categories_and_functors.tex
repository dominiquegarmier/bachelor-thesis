\chapter{Categories and Functors}
It is well known \source that there is no such thing as a set of all sets. However one might still want to reason about such collections. For this reason we define the notion of a class.

... for our purpose we only care about the fact that a class can contain mathematical objects...

\section{Categories}
Abstract notion of composition...

\begin{definition}{Category}{}
\cite{Roman2017} \source A category is a triple $(\ob, \mor, \circ)$ where:
\begin{enumerate}
    \item \textbf{Objects} $\ob$ is a class, elements of which we refer to as objects.
    
    \item \textbf{Morphisms} For every $A, B \in \ob$ we have a set $\mor(A,B)$. An element $f \in \mor(A,B)$ is called a morphism from $A$ to $B$ and we often write $f: A \to B$. Furthermore $\mor(A,B)$ and $\mor(C,D)$ are disjoint unless $A = C$ and $B = D$.
    
    \item \textbf{Composition} For every $f \in \mor(A,B)$ and $g \in \mor(B,C)$ there exists a $g \circ f \in \mor(A,C)$ called the composition of $g$ after $f$. And we have associativity: 
    $$
    f \circ (g \circ h) = (f \circ g) \circ h
    $$

    \item \textbf{Identity} For each object $A \in \ob$ there exists a morphism $e_A \in \mor(A,A)$ such that for $f \in \mor(A,B)$ and $g \in \mor(B,A)$ we have
    $$
    f \circ e_A = f \quad \text{and} \quad e_A \circ g = g
    $$
\end{enumerate}
\end{definition}

Morphisms can be empty and...

\begin{definition}{Subcategory}{}
\source

In what comes later, we will often consider a collection of categories sharing the same objects, however inclusions of morphism.
\end{definition}
\begin{myremark}{}{}
Morphisms can be empty and... Subcategories
\end{myremark}

... inutively morphism can often be thought of as maps...
as seen in this example

\begin{example}{Category of Sets}{}
Consider the category $\mathcal{C}_\mathrm{set}$ consisting of $\ob$ being the class containing all sets and $\mor(A,B) := B^A$ the set of maps from $A$ to $B$. In this case composition is simply the composition of maps and the identity is given by the identity map $e_A: A \ni x \mapsto x \in A$.
\end{example}

... categories can conserve structure.

\begin{example}{Category of Topological Spaces}{}
\end{example}

... morphisms must not always represent maps.

\begin{example}{Category of Ordering}{}
\end{example}

\section{Functors}
maps between categories that respect composition

\begin{definition}{Functor}{}
\end{definition}

\begin{example}{Natural Transformation}{}
\source
\end{example}

\begin{example}{Group Homomorphism as a Functor \todo}{}
Let $G$ be a group in the usual sense. Then we can construct a category $\mathcal{G}$ that represents the group. ... Take a single object $A$. Define $\mor(A,A) =  G$ with $f,g \in \mor(A,A): f\circ g = fg$ and $e_A = e$
\end{example}

\begin{example}{Homology Functor}{}
\end{example}