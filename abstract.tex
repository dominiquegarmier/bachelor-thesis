\begin{abstract}
Clustering algorithms are widely used in data science. \todo[expand this]

\textsc{Kleinberg} \cite{Kleinberg2002} has show that for certain seemingly reasonable assumptions no clustering algorithm can satisfy them all.

In this thesis we will review techniques used to formalize clustering algorithms developed by Carlsson and Memoli \cite{Carlsson2010}. This will give rise to the definition of the \emph{classical clustering functor} as well as the \emph{hierarchical clustering functor}.

We will introduce the Vietoris-Rips functor and build upon characterizations of it shown by \cite{Carlsson2010} to proof a new uniqueness result. Namely, every surjective regular clustering functor is in some sense the Vietoris-Rips functor.

Lastly \cite{Carlsson2010} remarked on how \textsc{Kleinberg}'s impossibility conditions can be rephrased as conditions for \emph{hierarchical clustering functors}. It turns out that under this reformulation our uniqueness result can be used to show that the Vietoris-Rips functor is the only clustering functor satisfying these \emph{modified Kleinberg conditions}.
\end{abstract}
