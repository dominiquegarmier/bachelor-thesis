\renewcommand{\abstractname}{Abstract}
\begin{abstract}
Clustering alogirthms are widely used in machine learning and data science as a tool for dealing with ever-increasing amounts of data.
An important theoretical result in the study of clustering algorithms is an impossibility theorem by Kleinberg \cite{Kleinberg2002}. It states that no clustering algorithm can be \emph{rich}, \emph{consistent} and \emph{scale invariant} at the same time. We will discuss these terms later.
%
We review a formalization and classification of clustering algorithms developed by \textsc{Carlsson} and \textsc{M\'emoli} \cite{Carlsson2010}. For such formal clustering algorithms we will use the term \emph{clustering functor}.
%
As identified by \textsc{Carlsson} and \textsc{M\'emoli}, the so-called \emph{Vietoris-Rips} clustering functor has some unique characterizing properties \cite[Thm.~18]{JMLR:v11:carlsson10a}, \cite[Thm.~7.1]{Carlsson2010}.
The Vietoris-Rips clustering functor is based on the idea that for some threshold parameter $\delta > 0$ we assign two data points to the same cluster if, according to some metric, they are $\delta$-close to each other.
We slightly generalized this characterization. It was previously shown that the Vietoris-Rips clustering functor satisfies a set of modified conditions from \textsc{Kleinberg}'s impossibility theorem \cite[Sec.~7.3.1]{Carlsson2010}. With our generalization we were able to show that the Vietoris-Rips clustering functor is the only clustering functor satisfying these conditions.
\end{abstract}