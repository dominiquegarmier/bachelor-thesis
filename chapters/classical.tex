\chapter{Classifying Classical Clustering Functors}

\begin{proposition}{Classification of $\iso$ Clustering Functors}{}
Let $\Cf: \iso \to \C$ be a clustering functor. Then there exists a choice ... moreover any such choice is a clustering functor.
\end{proposition}
In light of the previous proposition let now $\M \in \{\inj, \gen\}$


A very important functor is the:
\begin{definition}{Vietoris Rips Functor}{}
$\Rf_\delta$
\end{definition}

\todo[to what degree should I discuss the relevance of VR in other domains?]



\begin{proposition}{Vietoris Rips Functor}{}
$\Rf_\delta$ is an excessive and surjective clustering functor.
\end{proposition}

\section{Representable Clustering Functors}

\begin{definition}{Represented Clustering Functor}{}
Let $\Omega$ be a family of finite metric space. We define the clustering functor represented by $\Omega$
$$
\Cf^\Omega: \M \to \C
$$
with $\Cf(X,d) = (X, X/_\sim)$ where $\sim$ is generated by:
\begin{equation}
\label{eq:represented_clustering_functor_condition}
x \sim y \quad \text{if} \quad \exists \omega \in \Omega \,\exists \phi \in \mor_\M(\omega, X) \ \st \ \{x,y\} \subset \image(\phi)
\end{equation}
\end{definition}

$\Cf^\Omega$ is indeed a clustering functor. Let $f: X \to Y$ be a morphism and $x \sim y$. Then we have an $\omega \in \Omega$ and a $\phi \in \mor_\M(\omega, X)$ such that \ref{eq:represented_clustering_functor_condition} holds. But then $f \circ \phi \in \mor_\M(\omega,Y)$ has the same properties and it follows that $f(x) \sim f(y)$.

\begin{proposition}{Surjective Generative Model}{}
\todo[is this prop true?]
Let $\Omega$ be a generative model then the following are equivalent:
\begin{enumerate}
    \item $\Cf^\Omega$ is surjective
    \item $\exists \omega \in \Omega$ with $|\omega| = 2$ and for all $k \in \N$ we have 
    $$
    \sup\{\diam(\omega): \omega \in \Omega \text{ and } |\omega| = k\} < \infty.
    $$ 
\end{enumerate}
furthermore such a generative model $\Omega$ is called \emph{surjective}.
\end{proposition}





\begin{theorem}{Excessive Clustering Schemes}{}
Let $\M \in \{\inj, \gen\}$ then a clustering functor $\Cf: \M \to \C$ is representable if and only if it is excessive.
\end{theorem}

\begin{definition}{Generative Metric \cite{Carlsson2010}}{}
Let $\Omega$ be a generative model. Let $(X,d)$ be a metric space, we define a symmetric positive definite function $W^\Omega: X \times X \to [0,+\infty]$. With:
\begin{equation*}
W^\Omega(x,y) := \begin{cases}
    1/2 & 
    \begin{array}{cc}
         \text{if } \exists \, \omega \in \Omega \ \exists \, \phi \in \mor_{\M}(\omega, (X,d))\\
         \text{ such that } \{x,y\} \subset \image(\phi)
    \end{array}\\
    1 & \hfil \text{else}
\end{cases}
\end{equation*}
\end{definition}




\begin{myremark}{}{}
Let $d_{W^\Omega}$ be the shortest path distance metric \ref{def:shortest_dist_metric} then $(X,d) \mapsto (X,d_{W^\Omega})$ induces a functor $\Tf^\Omega: \M \to \M$.
\end{myremark}

\begin{theorem}{Factorizing Theorem}{}
Let $\Omega$ \footnote{\cite{Carlsson2010} provides a construction of $\Tf^\Omega$ that works for finite generative models.} be a generative model then we have
\begin{equation*}
\Cf^\Omega = \Rf_1 \circ \mathfrak{T}^\Omega
\end{equation*}
\end{theorem}

\section{Splitting Clustering Functors}

\begin{definition}{Splitting}{}
We say that a clustering functor $\Cf: \M \to \C$ is \emph{splitting} at $\delta_0 > 0$ if we have

\begin{enumerate}
    \item $\Cf(\Delta_2(\delta))$ is trivial for all $\delta < \delta_0$.
    \item $\Cf(\Delta_2(\delta))$ is discrete for all $\delta \geq \delta_0$.
\end{enumerate}

\todo[check inequalities, check with def of vietoris-rips]
\end{definition}

\begin{proposition}{}{splitting_inj}
Let $\Cf: \inj \to \C$ be a clustering functor that splits at $\delta > 0$. Then we have $\Rf_\delta \refines \Cf$.
\end{proposition}

\begin{proposition}{}{}
Let $\Cf: \gen \to \C$ be a clustering functor then $\Cf$ is splitting at $\delta > 0$ if and only if $\Rf_\delta = \Cf$.
\end{proposition}

\begin{proof}
To show $\Rf_\delta \refines \Cf$ we can adapt the proof of the previous proposition \ref{prop:splitting_inj}. It therefore remains to show $\Cf \refines \Rf_\delta$. \todo
\end{proof}

\section{Scaling Invariant Clustering Functors}

\begin{definition}{Scale Invariance}{scale_invariance}
A clustering functor $\Cf: \M \to \C$ is called \emph{scale invariant} if for all $\lambda > 0$ and $(X,d) \in \ob(\M)$ we have:
\begin{equation*}
    \Cf((X,d)) = \Cf(X,\lambda \cdot d)).
\end{equation*}
\end{definition}

\begin{proposition}{Scale Invariance in $\gen$}{}
Let $\Cf: \gen \to \C$ be a scale invariant functor then either
\begin{enumerate}
    \item $\Cf(X,d)$ is trivial for all $(X,d) \in \ob(\gen)$
    \item $\Cf(X,d)$ is discrete for all $(X,d) \in \ob(\gen)$
\end{enumerate}
\end{proposition}

\begin{proof}
$\Delta_2(\delta) \to X \to \Delta_2(\delta')$
\end{proof}

\begin{proposition}{Scale Invariance in $\inj$}{}
Let $\Cf: \inj \to \C$ be a scale invariant functor then...
\end{proposition}

\begin{proof}
$\Delta_{|X|}(\delta) \to X \to \Delta_{|X|}(\delta')$
\end{proof}