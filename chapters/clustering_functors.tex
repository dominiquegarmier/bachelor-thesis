\chapter{Clustering Functors}
\label{chapter__clustering_functor}

In this chapter we introduce the notion of a \emph{clustering functor}.
Ultimately, a clustering functor will be a functor, with certain properties, from a category of finite metric spaces to a category of partitions. Functoriality gives us a nice way of talking about certain invariances described in Section \ref{seciton__preserving_structure}.
This construction closely follows the one by \textsc{Carlsson} and \textsc{M\'emoli} \cite{Carlsson2010}.

\section{Finite Metric Spaces}
\label{section__finite_metric_spaces}

First we need to define the category of ``inputs'' of clustering algorithms. As we have seen before that it is natural to think of \emph{data} as a finite metric space. This motivates the following definition.

\begin{definition}{Finite Metric Spaces \cite[Sec.~3.2]{Carlsson2010}}{}
We define three categories $\iso, \inj$ and $\gen$, all sharing the same objects:
\begin{equation*}
\ob(\M) := \{(X,d): \text{$X$ a finite non-empty set and $d$ a metric on $X$}\}
\end{equation*}
for $\M \in \{\iso, \inj, \gen\}$.
The three categories are distinguished by their morphisms. For $A, B \in \ob(\M)$ we set:
\begin{itemize}
    \item $\mor_\gen(A,B)$ the distance non-increasing functions $f\colon A \to B$;
    \item $\mor_\inj(A,B)$ the distance non-increasing injective functions $f\colon A \to B$;
    \item $\mor_\iso(A,B)$ the isometries $f\colon A \to B$.
\end{itemize}
The composition of morphisms is given by the composition of functions, and the identity morphism is the identity function.
\end{definition}

Depending on what kind and the amount of ``structure'' we would like our clustering algorithms to ``preserve'', we can choose one of the three categories.
By construction, we have the inclusions
\begin{equation}
    \label{eq:inclusions_of_finite_metric_space_categories}
    \iso \subset \inj \subset \gen.
\end{equation}
Furthermore, this inclusion yields functors $\iso \to \inj \to \gen$ \cite[Def.~1.2.18]{Leinster2014-dc}.
As such, $\iso$ is the category with the fewest morphisms. Indeed, between most objects there are no morphisms.
All morphisms are between spaces of the same isometry class and have an inverse.
In contrast, $\gen$ has the most morphisms. In particular, for any metric spaces $A,B \in \ob(\gen)$ we have the morphism
$$
\mathrm{const}_b\colon A \to B, \ a \mapsto b
$$
for some $b \in B$.

\section{Partitions and Dendrograms}
\label{section__partitions_old}






At this point we need some additional structure that will allow us to define categories based on partitions.

\begin{definition}{}{}
Let $f\colon X \to Y$ be a map. For some partition $P \in \P(Y)$ we define $f^*(P) \in \P(X)$ to be the partition such that
\begin{equation*}
    \forall x,y \in X: x \sim_{f^*(P)} y \iff f(x) \sim_P f(y).
\end{equation*}
We call $f^*(P)$ the \emph{pullback partition of $P$ w.r.t. $f$}.
\end{definition}

\begin{example}{}{pullback}
Consider $X := \{a,b,c\}$, $Y := \{a,b\}$ and $f\colon X \to Y$ such that
$$
f(a) = a \quad \text{and} \quad f(b) = f(c) = b.
$$
If we have a partition
$$
Q := \{\{a\}, \{b\}\}
$$
of $Y$. Then $f^*(Q) = \{\{a\}, \{b,c\}\}$ is the pullback partition of $Q$ w.r.t $f$. Notice how $f^*(Q)$ consists of the preimages of the parts of $Q$. 
\end{example}


\begin{definition}{}{}
Moreover, if $f \colon X \to Y$ is a map and $Q \in \P(Y)$ then we write $P \refines_f Q$ if $P \refines f^*(Q)$ and say that $P$ \emph{refines} $Q$ \emph{via} $f$.
\end{definition}

\begin{example}{}{refinement_via}
Consider $X := \{a,b,c\}$ and $Y := \{a,b\}$ and $f\colon X \to Y$ as in Example \ref{exa:pullback}. If we have the partition 
$$
Q := \{\{a\}, \{b\}\}
$$
of $Y$ then we have already seen that $f^*(Q) = \{\{a\}, \{b,c\}\}$. Thus, for the partition
$$
P := \{\{a\}, \{b\}, \{c\}\}
$$
of $X$ we have $P \refines_f Q$.
\end{example}

A useful fact to remember is that if $\mathrm{id}\colon X \to X$ is the identity then
\begin{equation}
    \label{eq:refinement_via_identity}
    P \refines_{\mathrm{id}} Q \iff P \refines Q.
\end{equation}

We can finally define the categories which will be the ``outputs'' of clustering functors.

\begin{definition}{Classical Clustering Outputs\cite[Def.~3.2]{Carlsson2010}}{output_classical_clustering_functors}
The category $\C$ of \emph{classical clustering outputs} is defined by
\begin{equation*}
    \ob(\C) := \{(X, P): X \text{ finite non-empty and } P \in \P(X)\}
\end{equation*}
and for all $(X,P), (Y,Q) \in \ob(\C)$ we have the morphisms
\begin{equation*}
    \mor_\C((X,P), (Y,Q)) := \{f\colon X \to Y: P \refines_f Q\}.
\end{equation*}
In short, we write $(X,P) \refines_f (Y,Q)$ for such a morphism. As before, the composition is given by composition of maps and the identity is the identity map.
\end{definition}

\begin{definition}{Hierarchical Clustering Outputs\cite[Def.~3.3]{Carlsson2010}}{output_hierarchical_clustering_functors}
Similarly, we define the category $\H$ of \emph{hierarchical clustering outputs}, given by
\begin{equation*}
\begin{multlined}[c][.8\displaywidth]
    \ob(\H) := \Big\{(X, \theta_X): X \text{ finite non-empty} \\ \text{and } \theta_X: \R_{\geq0} \to \P(X) \ \text{a dendrogram}\Big\}
\end{multlined}
\end{equation*}
and for all $(X, \theta_X), (Y, \theta_Y) \in \ob(\H)$ we have the morphisms
\begin{equation*}
    \mor_\H((X, \theta_X), (Y,\theta_Y)) :=
    \left\{f\colon X \to Y: \forall r \in \R_{\geq0}: \theta_X(r) \refines_f \theta_Y(r)) \right\}.
\end{equation*}
Again, we write $(X, \theta_X) \refines_f (Y, \theta_Y)$ for such a morphism, and the composition and identity are defined as before.
\end{definition}

Another way of thinking about morphisms in $\C$ (or $\H$) is that given ${(X,P), (Y,Q) \in \ob(\C)}$ a morphism $f \in \mor_\C((X,P), (Y,Q))$ is simply a map $f\colon X \to Y$ such that
$$
\forall x,y \in X: x \sim_P y \implies f(x) \sim_Q f(y).
$$

\section{Clustering Functors}
We now have all the tools to define a \emph{clustering functor}.

\begin{definition}{Clustering Functor \cite[Sec.~4.1]{Carlsson2010}}{}
Let $\M \in \{\iso, \inj, \gen\}$ and $\A \in \{\C, \H\}$. An $\M$-\emph{functorial clustering functor} (or $\M$ \emph{clustering functor}) is a functor from $\M$ to $\A$
$$\Cf : \M \longrightarrow \A$$
such that $\Cf$ factorizes the forgetful functors (see \ref{not:factorizing_forgetful_functor}). If $\A = \C$ we say that $\Cf$ is \emph{classical} and otherwise if $\A = \H$ we say it is \emph{hierarchical}.
\end{definition}

We can express the functoriality of a clustering functor $\Cf$ by the following commutative diagram.
\begin{equation*}
    \begin{tikzcd}
    {(X,d)} \arrow[r, "f"] \arrow[d, "\Cf", Rightarrow] & {(Y,d)} \arrow[d, "\Cf", Rightarrow] \\
    {\Cf(X,d)} \arrow[r, "\refines_f"]                  & {\Cf(Y,d)}
    \end{tikzcd}
\end{equation*}

For a hierarchical clustering functor $\Hf$ we will use the simplified notation:
$$
\Hf(X,d;r) := (X,\theta_X(r)).
$$
Moreover, for any $r \in \R_{\geq0}$ a hierarchical clustering functor $\Hf$ induces a classical clustering functor $\Hf(\ \cdot \ ;r)$.

\begin{myremark}{}{induced_functor_by_inclusion}
Recall the inclusions \eqref{eq:inclusions_of_finite_metric_space_categories} and their induced functors.
Given a clustering functor on a larger category, say $\Cf\colon \gen \to \C$, this immediately induces clustering functors on the smaller categories by pre-composition with the inclusion functors $\iso \to \inj \to \gen$. We will use the same symbol for the induced functors.
\end{myremark}

With this in mind, it makes sense to think of the categories $\iso, \inj, \gen$ as being different levels of ``structure'' a clustering functor can ``preserve'' where $\iso$ is the least restrictive and $\gen$ the most restrictive.


\begin{myremark}{}{}
We can extend the partial order $\refines$ on $\P(X)$ to a partial order on clustering functors.
In particular, if $\Cf, \mathfrak{D}\colon \M \to \C$ are classical clustering functors we write $\Cf \refines \mathfrak{D}$ if
\begin{equation*}
    \forall (X,d) \in \ob(\M): \Cf(X,d) \refines \mathfrak{D}(X,d).
\end{equation*}
And in case of hierarchical clustering functors $\Cf, \mathfrak{D}\colon \M \to \H$ we write $\Cf \refines \mathfrak{D}$ if
\begin{equation*}
    \forall (X,d) \in \ob(\M) \, \forall r \in \R_{\geq0}: \Cf(X,d;r) \refines \mathfrak{D}(X,d;r).
\end{equation*}
\end{myremark}

\Needspace{8\baselineskip}

\subsection{The Vietoris-Rips Clustering Functor}

A very natural example of a clustering functor is the so-called \emph{Vietoris-Rips clustering functor}. It can be interpreted as both a classical and hierarchical clustering functor. The next two chapters will be dedicated to studying the unique properties of this clustering functor.

\begin{example}{Vietoris-Rips Functor \cite[Def.~6.1]{Carlsson2010}}{classical_vr}
    Let $\delta > 0$ and $\M \in \{\iso, \inj, \gen\}$. The \emph{classical Vietoris-Rips clustering functor}
    $$
    \Rf_\delta\colon \M \to \C
    $$
    assigns to each metric space $(X,d) \in \M$ the partition $(X,P)$ where $\sim_P$ is the equivalence relation generated by:
    \begin{equation}
        \label{eq:vietoris_rips_equivalence_relation}
        \forall x,y \in X: d(x,y) \leq \delta \implies x \sim_P y.
    \end{equation}

    Let us now show that $\Rf_\delta$ is indeed a clustering functor.
    For this, it is sufficient to show that $\Rf_\delta$ is $\gen$-functorial. By Remark \ref{rem:induced_functor_by_inclusion}, functoriality over $\iso$ and $\inj$ will follow.
    \medskip

    Let $(X,d),(Y,d') \in \ob(\gen)$ and $(X,P) := \Rf_\delta(X)$ as well as $(Y,Q) := \Rf_\delta(Y)$.
    Take any $f \in \mor_\gen(X,Y)$, recall that $f$ is distance non-increasing.
    We have to show that 
    $$P \refines_f Q.$$
    Indeed, let $x,y \in X$ such that $d(x,y) \leq \delta$.
    Then, we have
    $$
    d'(f(x), f(y)) \leq d(x,y) \leq \delta
    $$
    and therefore $f(x) \sim_{Q} f(y)$.
    By taking the transitive closure we get that $P \refines_f Q$, and we are done.
\end{example}

\begin{myremark}{}{}
One could use a proper inequality in \eqref{eq:vietoris_rips_equivalence_relation}. The entire theory we are going to present would still hold, provided we tweak certain definitions accordingly, in particular, Definition \ref{def:dendrogram}.
\end{myremark}

By varying the parameter $\delta$ we can obtain the hierarchical Vietoris-Rips clustering functor.

\begin{example}{Vietoris-Rips Functor \cite[Ex.~7.1]{Carlsson2010}}{hierarchical_vr}
For $\M \in \{\gen,\inj,\iso\}$ we can define the \emph{hierarchical Vietoris-Rips clustering functor} by
$$
\Rf\colon \M \to \H
$$
where for $(X,d) \in \ob(\M)$ and $\delta > 0$ we set
$$
\Rf(X,d; \delta) := \Rf_\delta(X,d).
$$
Notice that since $\Rf_\delta$ is a classical clustering functor and $\Rf_\delta \refines \Rf_{\delta'}$ for $\delta \leq \delta'$, $\Rf$ is indeed a hierarchical clustering functor.
\end{example}

To get an understanding for how the Vietoris-Rips functor works, consider the following concrete example.

\begin{example}{}{}
Consider the seven points $\{a,b,c,d,e,f,g\} \subset \R^2$ shown in the Figure \ref{fig:vietoris_rips_example}. $\Rf_\delta$ creates the clusters $\{a,b,c\}$ and $\{d,e,f,g\}$, drawn in red and blue.
\begin{center}
\begin{minipage}{\textwidth}
\centering
\begin{tikzpicture}[sloped]
    \draw[dashed,fill=mygreen,fill opacity=0.1](2.3, 4.8) node[circle,fill=black,inner sep=1pt,fill opacity=1]{} circle (1);
    \draw[dashed,fill=mygreen,fill opacity=0.1](3.2, 4.0) node[circle,fill=black,inner sep=1pt,fill opacity=1]{} circle (1);
    \draw[dashed,fill=mygreen,fill opacity=0.1](1.7, 3.4) node[circle,fill=black,inner sep=1pt,fill opacity=1]{} circle (1);

    \draw[dashed,fill=mygreen,fill opacity=0.1](0.7, 1.3) node[circle,fill=black,inner sep=1pt,fill opacity=1]{} circle (1);
    \draw[dashed,fill=mygreen,fill opacity=0.1](3.9, 1.3) node[circle,fill=black,inner sep=1pt,fill opacity=1]{} circle (1);
    \draw[dashed,fill=mygreen,fill opacity=0.1](2.1, 1.1) node[circle,fill=black,inner sep=1pt,fill opacity=1]{} circle (1);
    \draw[dashed,fill=mygreen,fill opacity=0.1](3.4, 0.3) node[circle,fill=black,inner sep=1pt,fill opacity=1]{} circle (1);

    \draw[<->, >=stealth, color=mypurple, thick, opacity=0.6] (2.3, 4.8) -- (2.3 - 0.985, 4.8 + 0.174) node[midway, above]{\tiny{$\delta / 2$}};
    %\draw[<->, >=stealth, color=mypurple, thick, opacity=0.4] (3.2, 4.0) -- (3.2 + 0.866, 4.0 - 0.5) node[midway, above]{\tiny{$\delta$}};
    %\draw[<->, >=stealth, color=mypurple, thick, opacity=0.4] (1.7, 3.4) -- (1.7 - 0.174, 3.4 - 0.985) node[midway, above]{\tiny{$\delta$}};

    %\draw[<->, >=stealth, color=mypurple, thick, opacity=0.4] (0.7, 1.3) -- (0.7 - 0.5, 1.3 + 0.866) node[midway, above]{\tiny{$\delta$}};
    %\draw[<->, >=stealth, color=mypurple, thick, opacity=0.4] (3.9, 1.3) -- (3.9 + 0.866, 1.3 + 0.5) node[midway, above]{\tiny{$\delta$}};
    %\draw[<->, >=stealth, color=mypurple, thick, opacity=0.4] (2.1, 1.1) -- (2.1 - 0, 1.1 - 1) node[midway, above]{\tiny{$\delta$}};
    %\draw[<->, >=stealth, color=mypurple, thick, opacity=0.4] (3.4, 0.3) -- (3.4 + 0.866, 0.3 - 0.5) node[midway, above]{\tiny{$\delta$}};

    \draw[myred!50!black, thick, opacity=0.9, fill=myred, fill opacity=0.25] (2.3, 4.8) -- (3.2, 4.0) -- (1.7, 3.4) -- cycle;
    \draw[myblue!50!black, thick, opacity=0.9, fill=myblue, fill opacity=0.25] (3.9, 1.3) -- (2.1, 1.1) -- (3.4, 0.3) -- cycle;
    \draw[myblue!50!black, thick, opacity=0.9] (0.7, 1.3) -- (2.1, 1.1);

    \node[above] at (2.3, 4.8) {$a$};
    \node[above] at (3.2, 4.0) {$b$};
    \node[left] at (1.7, 3.4) {$c$};

    \node[left] at (0.7, 1.3) {$d$};
    \node[right] at (3.9, 1.3) {$e$};
    \node[above] at (2.1, 1.1) {$f$};
    \node[below] at (3.4, 0.3) {$g$};
\end{tikzpicture}
\captionof{figure}{Clusters (red and blue) produced by the Vietoris-Rips functor $\Rf_\delta$ for on the points $\{a,b,c,d,e,f,g\} \subset \R^2$.}
\label{fig:vietoris_rips_example}
\end{minipage}
\end{center}
For clarity, we draw the circles with radius $\delta / 2$, \ie\ two points are in the same cluster if circles around them with radius $\delta / 2$ intersect.
\end{example}

% \begin{example}{\cite[Ex.~6.1]{Carlsson2010}}{}
% Given $(X,d) \in \ob(\inj)$ we define
% $$
% \delta_{(X,d)} := \begin{cases}
%     \hfil 1 & \text{if $|X| = 1$} \\
%     \left(\min\{d(x,y): x \neq y \text{ in } X\}\right)^{-1} & \text{else}
% \end{cases}.
% $$
% Consider now the equivalence relation $\sim_{(X,d)}$ of $X$ generated by:
% $$
% \forall x,y \in X: d(x,y) \leq \delta_{(X,d)} \implies x \sim_{(X,d)} y.
% $$
% Let $P_{(X,d)} \in \P(X)$ denote the corresponding partition of $X$.
% The resulting map $\Cf: \inj \to \C$ such that $\Cf(X,d) := (X, P_{(X,d)})$ is a classical clustering functor.

% \medskip
% Let us check functoriality for $f \in \mor_\inj((X,d),(Y,d'))$. Since $f$ is distance non-increasing we immediately have that $\delta_{(X,d)} \leq \delta_{(Y,d')}$. Take $x,y \in X$ with
% \begin{equation}
%     d(x,y) \leq \delta_{(X,d)}.
%     \label{eq:distance_condition_example}
% \end{equation}
% Again, since $f$ is distance non-increasing, we get that
% $$
% d'(f(x),f(y)) \leq d(x,y) \leq \delta_{(X,d)} \leq \delta_{(Y,d')}
% $$
% and therefore $f(x) \sim_{(Y,d')} f(y)$. By taking the transitive closure over the condition in \eqref{eq:distance_condition_example}, we get that $P_{(X,d)} \refines_f P_{(Y,d')}$.
% This shows that $\Cf$ is indeed a clustering functor.
% \end{example}

% \begin{example}{}{}
% Let $\M \in \{\iso, \inj, \gen\}$
% Consider a classical clustering functor $\Cf: \M \to \C$ and $R > 0$.
% Then, for a metric space ${(X,d) \in \ob(\M)}$ let $\theta_X: \R_{\geq0} \to \P(X)$ be the dendrogram given by
% $$
% \theta_X(r) :=
% \begin{cases}
%     \text{discrete partition of $X$}& \text{if $r = 0$} \\
%     \hfil \text{trivial partition of $X$}& \text{if $r \geq R$} \\
%     \hfil \Cf(X,d) & \text{else}
% \end{cases}
% $$
% for all $r \in \R_{\geq0}$. 
% This defines a hierarchical clustering functor $\Hf: \M \to \H$ by setting
% $$
% \Hf(X,d) := (X,\theta_X)
% $$
% for all $(X,d) \in \ob(\M)$.
% \end{example}
