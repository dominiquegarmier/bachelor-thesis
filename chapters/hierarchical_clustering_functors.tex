\chapter{Hierarchical Clustering Functors}

\begin{definition}{Surjective}{}
    A hierarchical clustering functor $\Hf$ is said to be \emph{surjective} if for every possible output of a hierarchical clustering functor \ie\ $(X, \theta_X) \in \ob(\H)$ there exists a metric on $X$ such that $\Hf(X,d) = (X, \theta_X)$.
\end{definition}

Notice given some surjective hierarchical clustering functor $\Hf$ then any $t \in \R_{\geq0}$ induces a surjective classical clustering functor $\Cf_t(X,d) := \Hf(X, d; t)$.


\todo[I need to find out if this is equivalent]
One could define an alternative notion of surjectivity by asking that the induced functor $\Cf_t$ is surjective for all $t \in \R_{\geq0}$.

\begin{definition}{Scaling}{}
A hierarchical clustering functor $\Hf: \H \to \H$ is \emph{scaling} if for all $(X,d) \in \ob(\M)$ we have
$$
\Hf(X, \lambda d) = \sigma_\lambda \Hf(X,d)
$$
\todo[define notation earlier]
\end{definition}
This is an alternative to scale invariance.


\begin{definition}{Scaling Extension}{scale_invar_ext}
Let $\Cf: \M \to \C$ a classical clustering functor then we can \emph{extend} it to an almost unique scale invariant hierarchical functor
$$
\Hf_\Cf: \M \to \H
$$
\end{definition}


\begin{myremark}{Modified Kleinberg Conditions}{}
    \cite[Comment~7.3.1]{Carlsson2010}

    \todo[write this up]

    \begin{itemize}
        \item consistency becomes $\gen$ functoriality
        \item richness becomes surjectivity
        \item scale invariance becomes scaling
    \end{itemize}
\end{myremark}

\begin{theorem}{Uniqueness of Kleinberg's Conditions}{}
    Let $\Hf: \gen \to \H$ be a hierarchical clustering functor fulfilling the modified Kleinberg conditions then

    $$
    \Hf = \sigma_\delta \Rf
    $$

    \todo[define $s_\delta$]
    \newresult[check this]
\end{theorem}

\todo[it might be possible to show the above theorem even for $\inj$]