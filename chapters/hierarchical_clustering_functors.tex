\chapter{Hierarchical Clustering Functors}
\section{Scaling}

\begin{definition}{Scaling}{}
We say that a hierarchical clustering functor $\Hf: \M \to \H$ is scaling if ...
\end{definition}

\begin{definition}{Scaling Extension}{scale_invar_ext}
Let $\Cf: \M \to \C$ a classical clustering functor then we can \emph{extend} it to an almost unique scale invariant hierarchical functor
$$
\Hf_\Cf: \M \to \H
$$
\end{definition}

\section{Surjectivity}
\todo[some notes...]
Surjectivity in (scaling) hierarchical clustering functors might be stronger, we need to check this.
If it turns out to be stronger, I might be able to prove the next theorem in more generality.

\section{Kleinberg's Conditions}

\begin{myremark}{Modified Kleinberg Conditions}{}
    \cite[Comment~7.3.1]{Carlsson2010}

    \todo[write this up]

    \begin{itemize}
        \item consistency becomes $\gen$ functoriality
        \item richness becomes surjectivity
        \item scale invariance becomes scaling
    \end{itemize}
\end{myremark}

\begin{theorem}{Uniqueness of Kleinberg's Conditions}{}
    Let $\Hf: \gen \to \H$ be a hierarchical clustering functor fulfilling the modified Kleinberg conditions then

    $$
    \Hf = s_\delta \Rf
    $$

    \todo[define $s_\delta$]
    \newresult[check this]
\end{theorem}

\todo[it might be possible to show the above theorem even for $\inj$]