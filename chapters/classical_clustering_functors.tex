\chapter{Classification of Classical Clustering Functors}
\label{chapter__classical}
In this chapter any clustering functor is of the classical type.

An important (classical) clustering functor is the Vietoris-Rips functor.
In particular, we will later see that it exhibits certain unique properties.

\begin{defprop}{Vietoris-Rips Functor}{classical_vr}
    Let $\delta > 0$ and $\M \in \{\iso, \inj, \gen\}$. The Vietoris Rips functor $\Rf_\delta: \M \to \C$ assigns to each metric space $(X,d) \in \M$ the partition $(X,P)$ where $\sim_P$ is the equivalence generated by:
    \begin{equation}
        \label{eq:vietoris_rips_equivalence_relation}
        \forall x,y \in X: d(x,y) \leq \delta \implies x \sim_P y
    \end{equation}
    \todo[check left/right regularity; this is the single source of truth]
\end{defprop}

One could use a proper inequality in \eqref{eq:vietoris_rips_equivalence_relation}. The entire theory we are about to present would still hold, provided we tweak certain definitions accordingly in particular \ref{def:dendogram}.

\begin{proof}
It is sufficient to show that $\Rf_\delta$ is $\gen$ factorial by remark \ref{rem:induced_functor_by_inclusion} functoriality over $\iso$ and $\inj$ will follow.

Let $X,Y \in \ob(\gen)$ and $(X,P_X) := \Rf(X)$ as well as $(Y,P_Y) := \Rf(Y)$.
Take any $f \in \mor_\gen(X,Y)$, recall that $f$ is distance non-increasing.
We have to show that 
$$P_X \refines f_*(P_Y).$$
Indeed, let $x,y \in X$ such that $d(x,y) \leq \delta$.
Then we have $d(f(x), f(y)) \leq d(x,y) \leq \delta$ and therefore $f(x) \sim_{P_Y} f(y)$.
By taking the transitive closure we get the statement.
\end{proof}

\begin{myremark}{$\iso$ Clustering Functors \cite{Carlsson2010}}{}
Let $\mathcal{I}$ denote a collection of representatives of isometry classes of finite metric spaces.
For each $X \in \mathcal{I}$ consider the isometry group $\mathrm{Iso}(X)$. \par

\medskip This group acts on $\mathfrak{P}(X)$ via $(\phi, P) \mapsto \phi_*(P)$.
Let $\Xi_X \subseteq \mathfrak{P}(X)$ denote the set of fixed points of this group action. \par

\medskip A clustering functor $\Cf: \iso \to \C$ is uniquely determined by a choice of $P_X \in \Xi_X$ for each $X \in \mathcal{I}$.
\end{myremark}

Clearly $\iso$ permits many clustering functors.
In some sense $\iso$ clustering functors can be thought of as any \emph{algorithm} which \emph{does not consider any arbitrary ordering} of the data points.
This is the reason that algorithms like single linkage clustering \ref{section__linkage_clustering} is not even $\iso$ functorial.

From now on, unless otherwise mentioned, we use $\M \in \{\inj, \gen\}$.

\section{Excessive and Representable Clustering Functors}
\begin{definition}{Excessive Clustering Functors \cite{Carlsson2010}}{}
A clustering functor $\Cf: \M \to \C$ is called \emph{excessive} if for every $(X,d) \in \ob(\M)$ and $(X,P) = \Cf(X,d)$ we have that for every block $X_\alpha \in P$:
$$
\Cf(X_\alpha, d|_{X_\alpha \times X_\alpha}) = (X, \{X_\alpha\})
$$
\end{definition}

Let $\Omega$ be a family of finite non-empty metric spaces. Such $\Omega$'s can be used to construct clustering functors. We think of $\Omega$ as being a collection of \emph{patters} and our clustering algorithm as detecting these \emph{patterns}.



\begin{definition}{Representable Clustering Functors \cite{Carlsson2010}}{construction_of_representable_clustering_functors}
We define the clustering functor \emph{represented by $\Omega$}
$$
\Cf^\Omega \M \to \C
$$
where $\Cf(X,d) = (X, P)$ such that $\sim_P$ 
is the equivalence relation generated by:
\begin{equation*}
    \forall x,y \in X: \ \exists \omega \in \Omega \,\exists \phi \in \mor_\M(\omega, (X,d)) \ \st \ \{x,y\} \subset \image(\phi) \implies x \sim_P y.
\end{equation*}
Additionally, $\Cf$ is said to be representable if there exists some $\Omega$ such that $\Cf = \Cf^\Omega$.
\end{definition}

This definition is valid for any $\Omega$, but we will see that imposing certain conditions on $\Omega$ will yield interesting properties of $\Cf^\Omega$.

\begin{myremark}{}{}
Notice that for $(X,d) \in \Omega$ we have $\Cf^\Omega(X,d) = (X, \{X\})$.
\end{myremark}

This leads to an alternative characterization of $\Cf^\Omega$.

\begin{proposition}{}{}
For any clustering functor $\Cf$ such that $\Cf(X,d)$ is trivial for all $(X,d) \in \Omega$ we have
$$
\Cf^\Omega \refines \Cf
$$
\end{proposition}

\begin{proof}
Let $(X,d) \in \ob(\M)$ and $(X,P) = \Cf^\Omega(X,d)$ and $(X,Q) = \Cf(X,d)$. We want to show that $P \refines Q$.

Let $x,y \in X$ such that the generating condition of $P$ holds. \Ie\ there exists $\omega \in \Omega$ and $\phi \in \mor_\M(\omega, (X,d))$ such that $\{x,y\} \subset \image(\phi)$.

But then since $\Cf(\omega)$ is trivial by assumption and by functoriality $\Cf(\omega) \refines f_*(\Cf(X,d))$ we get that $x \sim_Q y$. Taking the transitive closure gives the statement.


\newresult[check this]
\end{proof}

Importantly $\Cf^\Omega$ is finest clustering functor such that $\Cf^\Omega(X,d)$ is trivial for all $(X,d) \in \Omega$. The existence of this minimal clustering functor follows by our initial construction \ref{def:construction_of_representable_clustering_functors}.

\begin{theorem}{\cite{Carlsson2010}}{}
Let $\M \in \{\inj, \gen\}$. A clustering functor $\Cf: \M \to \C$ is representable if and only if it is excessive.
\end{theorem}

\begin{proof}
First we show that $\Cf^\Omega$ is excessive. So let $(X,d) \in \ob(\M)$ and $(X,P) = \Cf^\Omega(X,d)$ with $X_\alpha \in P$.

Let $x,y \in X_\alpha$ such that there exists $\omega \in \Omega$ and $\phi \in \mor_\M(\omega, (X,d))$ such that $\{x,y\} \subset \image(\phi)$. But since the injection $\iota: X_\alpha \to X$ is injective and distance non-increasing we get that $\iota \circ \phi \in \mor_\M(\omega, (X_\alpha, d|_{X_\alpha \times X_\alpha}))$. By taking the transitive closure we conclude that $\Cf^\Omega(X_\alpha, d|_{X_\alpha \times X_\alpha}) = (X, \{X_\alpha\})$ and $\Cf^\Omega$ is excessive.

It remains to show that any excessive clustering functor is representable. For this consider
$$
\Omega := \{(X_\alpha, d|_{X_\alpha \times X_\alpha}): (X, \{X_\alpha\}_\alpha) = \Cf(X,d) \text{ for } (X,d) \in \ob(\M)\}.
$$
\todo[finish this]
\end{proof}

% \subsection{Factorization Theorem}
% \todo[this subsection is kind of irrelevant]

% \begin{definition}{Bounded Generative Model}{}
% We say that $\Omega$ is \emph{bounded} if for all $k \in \N$ we have
% $$
% \sup\{\diam(X): X \in \Omega \text{ and } |X| = k\} < \infty
% $$
% \end{definition}
% Notice that if $|\Omega| < \infty$ then $\Omega$ is clearly bounded.

% \begin{definition}{Generative Metric}{}
% \end{definition}

% \begin{theorem}{}{}
%     Factorization Theorem
% \end{theorem}

\section{Surjective Clustering Functors}

\begin{definition}{Surjective Clustering Functors \source}{}
    A classical clustering functor $\Cf: \M \to \C$ is called \emph{surjective} if for every finite set $X$ and every $P_X \in \P(X)$ there exists a metric $d_X$ on $X$ such that

    \begin{equation*}
        \Cf(X,d_X) = (X,P_X)
    \end{equation*}

\end{definition}

\begin{myremark}{}{vietoris_rips_is_surjective}
    The Vietoris-Rips functor $\Rf_\delta$ is surjective.
\end{myremark}
For this consider any finite set $X$ and partition $P$ of $X$. We can now define the metric $d$ on $X$ by
\begin{equation*}
    d(x,y) := \begin{cases}
        0 & \text{if } x = y\\
        \delta & x \neq y \text{ and } x \sim_P y\\
        2\delta & \text{otherwise}
    \end{cases}.
\end{equation*}
It is easy to check that $\Rf_\delta(X,d) = (X,P)$.
The following property of surjective clustering functors hints that surjectivity and the Vietoris-Rips functor are closely related.

% \todo[this is outdated]
% The important property of the Vietoris rips functor that we used in the previous factorization theorem was that it is surjective. We can actually express this in a more general way:
% \begin{proposition}{}{}
%     Let $\mathfrak{S}: \M \to \C$ be a surjective functor. Then for any clustering functor $\Cf: \M \to \C$ there exists a functor $\Tf_\Cf: \M \to \M$ that changes the metric such that
%     \begin{equation*}
%         \Cf = \mathfrak{S} \circ \Tf_\Cf
%     \end{equation*}
%     \newresult \todo[remove this?]
% \end{proposition}
% \begin{proof]
%     \newresult[]
% \end{proof}

\begin{lemma}{}{surjective_implies_eventually_discrete}
    Let $\Cf: \M \to \C$ be a surjective clustering functor and $(X,d)$ a finite metric space. Then there exists $\lambda_0,\lambda_1 > 0$ such that
    \begin{enumerate}
        \item $\Cf(X,\mu \cdot d)$ is trivial for all $0 < \lambda \le \lambda_0$
        \item $\Cf(X,\lambda \cdot d)$ is discrete for all $\lambda \ge \lambda_1$
    \end{enumerate}

    \newresult[I need to check this again]
\end{lemma}

\begin{proof}
    Let $(X,d) \in \ob(\M)$ since $\Cf$ is surjective there exists metrics $d_0, d_1$ on $X$ such that $\Cf(X,d_0)$ is trivial and $\Cf(X,d_1)$ is discrete.

    \begin{enumerate}
        \item We take
        $$
        \lambda_0 := \frac{\sep(X,d_0)}{\diam(X,d)}.
        $$ 
        Notice that for any $\lambda \le \lambda_0$ we have that $\diam(X,\lambda \cdot d) \le \sep(X,d_0)$.
        Because of this the function
        \begin{align*}
            f: (X, d_0) &\longrightarrow (X, \lambda \cdot d)\\
            x &\longmapsto x
        \end{align*}
        is distance non-increasing (and injective) \ie\ $f \in \mor_\M((X,d_0), (X,\lambda \cdot d))$. Moreover, since $f$ is the identity on the set $X$ we have $f_*(P) = P$ for any partition $P$ of $X$.

        By functoriality of $\Cf$ we now have that $\Cf(X,d_0) \refines f_*(\Cf(X,\lambda \cdot d)) = \Cf(X,\lambda \cdot d)$. Therefore, $\Cf(X,\lambda \cdot d)$ must be trivial.

        \item Repeat an analogous argument using
        $$
        \lambda_1 := \frac{\diam(X,d_1)}{\sep(X,d)}.
        $$
        But this time consider for $\lambda \ge \lambda_1$ the function
        \begin{align*}
            f: (X, \lambda \cdot d) &\longrightarrow (X, d_1)\\
            x &\longmapsto x.
        \end{align*}
        We conclude by showing $\Cf(X,\lambda \cdot d) \refines f_*(\Cf(X,d_1)) = \Cf(X,d_1)$. 
        
        \newresult[check this]
    \end{enumerate}
\end{proof}

\section{Splitting}

A central property of the Vietoris-Rips functor is its behavior with respect to the two point space $\Delta_2(\lambda)$. In particular, we have that $\Rf_\delta(\Delta_2(\lambda))$ is trivial if and only if $\lambda \le \delta$ and discrete otherwise.
This motivates the following definitions.

\begin{definition}{}{}
    A clustering functor $\Cf: \M \to \C$ is called \emph{splitting} at $\delta_0 > 0$ if we have
    \begin{enumerate}
        \item $\Cf(\Delta_2(\delta))$ is trivial for all $\delta \leq \delta_0$.
        \item $\Cf(\Delta_2(\delta))$ is discrete for all $\delta > \delta_0$.
    \end{enumerate}
\end{definition}



The following results show that what extend splitting is a characterizing property of the Vietoris-Rips functor \cite[Theorem~6.4]{Carlsson2010}.

\begin{proposition}{}{splitting_vietoris_inj}
    Let $\Cf: \inj \to \C$ be a clustering functor that splits at $\delta > 0$. Then we have 
    $$
    \Rf_\delta \refines \Cf.
    $$
\end{proposition}


\begin{proposition}{\source}{splitting_vietoris_gen}
    Let $\Cf: \gen \to \C$ be a clustering functor then $\Cf$ is splitting at $\delta > 0$ if and only if 
    $$
    \Cf = \Rf_\delta
    $$
\end{proposition}

\begin{proof}[Proof of Proposition \ref{prop:splitting_vietoris_inj}]
    Let $(X,d) \in \ob(\inj)$ with $(X,P) = \Cf(X,d)$ and $x \neq y$ in $X$ with $d(x,y) \leq \delta$.
    Then there is a map $f \in \mor_\inj(\Delta_2(\delta), (X,d))$ with $\{x,y\} = \image(f)$.
    Since $\Cf(\Delta_2(\delta))$ is trivial we have by functoriality that $x \sim_P y$

    Taking the transitive closure we get that $\Rf_\delta(X,d) \refines P$ and the statement follows by generality of $(X,d)$.
    \source[proof from paper]
\end{proof}

\begin{proof}[Proof of Proposition \ref{prop:splitting_vietoris_gen}]
    \begin{description}
        \item[$\impliedby$:] Clearly $\Rf_\delta$ is splitting at $\delta$.
        \item[$\implies$:] Let $\Cf$ be splitting at $\delta$. In view of the previous proposition it remains to show that $\Cf \refines \Rf_\delta$.
        
        Let $(X,d) \in \ob(\gen)$ with $(X,P) = \Cf(X,d)$ and $(X,R) = \Rf_\delta(X,d)$. We will show that $P \refines R$, more precisely we show that for all $x,y \in X$ we have $x \not\sim_R y \implies x \not\sim_P y$. For this we define
        $$
        \delta_0 := \min\{d(x,y): x,y \in X \text{ and } x \not\sim_R y\}
        $$
        Let now $x,y \in X$ such that $x \not\sim_R y$.
        Then we can find a map $f \in \mor_\gen((X,d), \Delta_2(\delta_0))$ such that $f(x) \neq f(y)$\footnote{One particular way to construct such an $f$ would be to send every $x' \sim_R x$ to one point in $\Delta_2(\delta_0)$ and every $x' \not\sim_R x$ to the other point. By the way we defined $\delta_0$ this map would in fact be distance non-increasing.}.

        Since $\delta_0 > \delta$ and $\Cf$ is splitting we get $f(x) \not\sim f(y)$ so by functoriality we have $x \not\sim_P y$.
    \end{description}
    \source[proof from paper]
\end{proof}

\begin{example}{}{}
In some sense the two previous propositions are the best we can get. For $\delta>0$ consider the clustering functor $\bar\Rf_\delta: \inj \to \C$ defined by
$$
\bar\Rf_\delta(X,d) := \Rf_{\frac{\delta|X|}{2}}(X,d).
$$
Moreover, $\bar\Rf_\delta$ is splitting at $\delta$.

\medskip
To see that $\bar\Rf_\delta$ is indeed a clustering functor consider metric spaces $(X,d), (Y,d') \in \ob(\inj)$ and a morphism $f \in \mor_\inj((X,d), (Y,d'))$. By injectivity of $f$ we immediately get that $|X| \le |Y|$. And we have
$$
\bar\Rf_\delta(X,d) = \Rf_{\frac{\delta|X|}{2}}(X,d) \overset{(1)}\refines \Rf_{\frac{\delta|Y|}{2}}(X,d) \overset{(2)}\refines f_*\Rf_{\frac{\delta|Y|}{2}}(Y,d') = f_*\bar\Rf_\delta(Y,d').
$$
Where $(1)$ follows from the fact that $|X| \le |Y|$ and $(2)$ follows from the fact that $\Rf_\delta'$ is $\inj$ functorial for any $\delta' > 0$. Therefore, $\bar\Rf_\delta$ is an $\inj$ clustering functor.

\medskip
Using a similar argument to the one we used to show that $\Rf_\delta'$ is surjective we can also show that $\bar\Rf_\delta$ is surjective.

\newresult[this is a new construction, check it]
\end{example}

\section[Vietoris-Rips Uniqueness]{Vietoris-Rips Uniqueness \newresult[]}

As previously mentioned the Vietoris-Rips functor is surjective.
In this section we present the new result that under certain technical assumptions the Vietoris-Rips is the unique surjective $\gen$ clustering functor.

We will later \todo[proper ref] \ref{thm:kleinbergs_uniqueness_conditions} discuss the implications this has on \textsc{Kleinberg}'s theorem \ref{thm:kleinberg}. 

\begin{definition}{}{}
    A clustering functor $\Cf: \M \to \C$ is \emph{regular} if for all $(X,d) \in \ob(\M)$ there exists some $\lambda \in (0, 1)$ such that
    $$
    \Cf(X,d) = \Cf(X, \lambda \cdot d).
    $$
    
    \newresult[new def, make sure this makes sense]
\end{definition}

Notice that $\Rf_\delta$ is regular. The reason we ask for $\lambda \in (0,1)$ and not $\lambda \in (1, \infty)$ is that this corresponds to the regularity condition \ref{rem:left_regularity_of_dendograms} for dendograms.\par

\begin{myremark}{}{left_regularity_classical_clusteirng_functors}
    \medskip It might seem overly restrictive to ask for regularity.
    However, if we take some $(X, d) \in \ob(\M)$ and consider the function 
    $$
    \lambda \mapsto \Cf(X, \lambda \cdot d),
    $$
    which will be piecewise constant and takes only finitely many values.
    Regularity ensures that this function is constant on $(a, b]$ for some $a < b$.
\end{myremark}

Assuming regularity and building on lemma \ref{lem:surjective_implies_eventually_discrete} we can now finally show that the Vietoris-Rips functor is the only surjective regular $\gen$ clustering functor.

\begin{theorem}{}{uniqueness_of_surjective_regular_functors}
    Let $\Cf: \gen \to \C$ then the following are equivalent

    \begin{itemize}
        \item $\Cf$ is surjective and regular
        \item $\Cf = \Rf_{\delta}$ for some $\delta > 0$
    \end{itemize}
    \newresult[make sure this makes sense]
\end{theorem}

\begin{lemma}{}{surjective_implies_splitting}
    Let $\Cf: \M \to \C$ be a surjective left-regular clustering functor. Then $\Cf$ is splitting at some $\delta > 0$.

    \newresult[make sure this makes sense]
\end{lemma}

\begin{proof}[Proof of Lemma]
    By lemma \ref{lem:surjective_implies_eventually_discrete} we know that there exists $\lambda_0, \lambda_1 > 0$ such that for all $\lambda \le \lambda_0$ we have $\Cf(X, \lambda \cdot d)$ is trivial and for all $\lambda' \ge \lambda_1$ we have $\Cf(X, \lambda' \cdot d)$ is discrete.

    Let now $\lambda \ge \lambda'$ by functoriality and considering the distance non-increasing function
    \begin{align*}
        f: (X, \lambda \cdot d) &\longrightarrow (X, \lambda' \cdot d)\\
        x &\longmapsto x
    \end{align*}
    we have that $\Cf(X, \lambda \cdot d) \refines f_*(\Cf(X, \lambda' \cdot d)) = \Cf(X, \lambda' \cdot d)$. $\star$

    If we now use $(X,d) = \Delta_2(1)$ we have that the map $\lambda \mapsto \Cf(\Delta_2(\lambda))$ is piecewise constant and can take at most two values.

    Thus, by $\star$ there must exist some $\delta_0 > 0$ such that for $\delta < \delta_0 < \delta'$ we have that $\Cf(\Delta_2(\delta))$ is trivial and $\Cf(\Delta_2(\delta'))$ is discrete.
    As for the value at $\delta_0$ we recall the previous remark \ref{rem:left_regularity_classical_clusteirng_functors} and conclude that $\Cf(\Delta_2(\delta_0))$ is trivial.
    \newresult[check this]
\end{proof}

\begin{proof}[Proof of Theorem]
    For the first implication recall that we have already discussed that $\Rf_\delta$ is surjective (see: \ref{rem:vietoris_rips_is_surjective}) and also regular.

    Assuming that $\Cf$ is left-regular and surjective, from the above lemma we get that $\Cf$ is splitting at some $\delta > 0$ so by \ref{prop:splitting_vietoris_gen} we have that $\Cf = \Rf_{\delta}$.
\end{proof}

\section{Scale Invariant Clustering Functors}

We will quickly talk about scale invariance inspired by \textsc{Kleinberg}'s conditions from theorem \ref{thm:kleinberg}, and see where this fails. \source

\begin{definition}{Scale Invariance}{scale_invariance}
    A clustering functor $\Cf: \M \to \C$ is called \emph{scale invariant} if for all $\lambda > 0$ and $(X,d) \in \ob(\M)$ we have:
    \begin{equation*}
        \Cf((X,d)) = \Cf(X,\lambda \cdot d)).
    \end{equation*}
\end{definition}

We now get two interesting distinct behavior of scale invariant clustering functors. First we consider the case of $\gen$.

\begin{proposition}{}{}
    Let $\Cf: \gen \to \C$ be a scale invariant clustering functor then either
    \begin{enumerate}
        \item $\Cf(X,d)$ is trivial for all $(X,d) \in \ob(\gen)$
        \item $\Cf(X,d)$ is discrete for all $(X,d) \in \ob(\gen)$
    \end{enumerate}
\end{proposition}

\begin{proof}
    \source[]
    By scale invariance we either have that $\Cf(\Delta_2(\delta))$ is trivial or discrete for all $\delta > 0$.

    Let $|X| \ge 2$ otherwise the statement is clear. Then for any $x \neq x'$ in $X$ we can find $\delta, \delta' > 0$ and morphisms in $\gen$
    $$
    \Delta_2(\delta) \overset{f}\longrightarrow X \overset{g}\longrightarrow \Delta_2(\delta')
    $$
    such that $\image(f) = \{x, x'\}$. By functoriality we get
    $$
    \Cf(\Delta_2(\delta)) \refines f_*\Cf(X,d) \refines \Cf(\Delta_2(\delta')).
    $$
    And since $x \neq x'$ was arbitrary in $X$ we get that if
    \begin{itemize}
        \item $\Cf(\Delta_2(\delta))$ is trivial then $\Cf(X)$ is trivial.
        \item $\Cf(\Delta_2(\delta))$ is discrete then $\Cf(X)$ is discrete.
    \end{itemize}
\end{proof}

In case of $\inj$ we have more interesting behavior.

\begin{proposition}{Scale Invariance in $\inj$}{}
    Let $\Cf: \inj \to \C$ be a scale invariant functor then there exists a $k \in \N \sqcup \{0, \infty\}$ such that for all $(X,d) \in \ob(\inj)$:

    \begin{itemize}
        \item If $|X| > k$ then $\mathfrak{C}(X,d)$ is trivial.
        \item If $|X| \le k$ then $\mathfrak{C}(X,d)$ is discrete.
    \end{itemize}

    Notice that if $k \in \{0, \infty\}$ then we recover the behavior from $\gen$.
\end{proposition}

\begin{proof}
    \source[]
    Notice that for $n \le n'$ there exists a morphism in $\inj$
    $$
    \Delta_n(\delta) \longrightarrow \Delta_{n'}(\delta)
    $$
    therefore we get that
    $$
    \Cf(\Delta_n(\delta)) \refines \Cf(\Delta_{n'}(\delta)).
    $$
    On the other hand any permutation of $\Delta_n(\delta)$ is also a morphism in $\inj$, this gives that $\Cf(\Delta_n(\delta))$ is either discrete or trivial.

    So together we get that there exists some $k \in \N \sqcup \{0, \infty\}$ such that
    \begin{itemize}
        \item $\forall n > k: \Cf(\Delta_n(\delta))$ is trivial.
        \item $\forall n \le k: \Cf(\Delta_n(\delta))$ is discrete.
    \end{itemize}
    By scale invariance this does not depend on $\delta$. We can now repeat the same argument as in the previous proposition to get the statement\footnote{By using $n = |X|$ we can find morphisms $f,g$ in $\inj$.}.

\end{proof}

It is immediately clear that scale invariant clustering functors can not be surjective.

% % new layout above ^^^
% \chapter*{Old classical clustering functors}
% % old layout below vvv


% \section{Elementary Properties}
% \subsection{Excessive Clustering Functors}
% \begin{definition}{Excessive Clustering Scheme}{}
% A classical clustering functor $\Cf: \M \to \C$ is called \emph{excessive} if for every $(X,d) \in \ob(\M)$ and $(X,P_X) = \Cf(X,d)$ we have that for every $X_\alpha \in P_X$
% \begin{equation*}
%     \Cf(X_\alpha, d|_{X_\alpha \times X_\alpha}) = (X,\{X_\alpha\}).
% \end{equation*}
% \end{definition}
% In other words if $\Cf$ is excessive it can be thought of as being nilpotent in a certain sense.

% \begin{myremark}{Excessive Clustering Algorithms}{}
% The notion of excessiveness exists in the more general setting of clustering alogrithms (which are not necessarily clustering functors).
% \end{myremark}

% \todo[reference to first chapter]
% In particular there we have encountered examples of non excessive clustering algorithms.

% \todo[explain why agglomerative / kmeans ... clustering is non excessive]

% \todo[give an example of a excessive clustering that we have seen before]

% \subsection{Surjective Clustering Functors}
% \begin{definition}{Surjective Clustering Functor}{}
% A classical clustering functor $\Cf: \M \to \C$ is called \emph{surjective} if for every finite set $X$ and every $P_X \in \P(X)$ there exists a metric $d_X$ on $X$ such that
% \begin{equation*}
%     \Cf(X,d_X) = (X,P_X)
% \end{equation*}
% \end{definition}

% We have seen this concept previously when discussing \textsc{Kleinberg}'s theorem \ref{thm:kleinberg}, there it was called richness \ref{def:richness}.

% As previously mentioned the $k$-means clustering algorithm\ref{def:kmeans_clustering} is clearly not surjective. As it limits the number of clusters.
% \todo[better wording]

% We will see later that surjective clustering functors are in some cases unique. Another relatively straightforward property of surjective functors is that they factor any other clustering functor:

% \begin{proposition}{}{universality_of_surjective_functors}
% Let $\mathfrak{S}: \M \to \C$ be a surjective functor. Then for any clustering functor $\Cf: \M \to \C$ there exists a functor $\Tf_\Cf: \M \to \M$ that changes the metric such that

% \begin{equation*}
% \Cf = \mathfrak{S} \circ \Tf_\Cf
% \end{equation*}
% \end{proposition}

% \mynote[move below to chapter 7?]
% A surprising fact about surjective clustering functors is given by the following lemma:

% \begin{lemma}{}{surjective_implies_eventually_discrete}
% Let $\Cf: \M \to \C$ be a surjective clustering functor and $(X,d)$ a finite metric space. Then there exists $\lambda_0, \mu_0 > 0$ such that
% \begin{enumerate}
%     \item $\Cf(X,\lambda \cdot d)$ is discrete for all $\lambda > \lambda_0$
%     \item $\Cf(X,\mu \cdot d)$ is trivial for all $0 < \mu \le \mu_0$
% \end{enumerate}
% \mynote[dedupe this from chapter 7]
% \end{lemma}

% Indeed this now motivates us to define the smallest resp. largest values for which the condition in the lemma holds. This gives us:

% \begin{definition}{}{}
% Let $\Cf: \M \to \C$ be surjective clustering functor. Due to the previous lemma we can define
% \begin{align*}
% \diam^\Cf(X) &:= \sup\{ \lambda > 0: \Cf(X,\lambda \cdot d) \text{ is trivial}\} < \infty\\
% \sep^\Cf(X) &:= \inf\{ \lambda > 0: \Cf(X,\lambda \cdot d) \text{ is discrete}\} < \infty
% \end{align*}
% called the \emph{intrinsic separation} and \emph{diameter} of $X$. 
% \end{definition}
% Additionally, it is easy to check that we can use $\lambda_0 = \sep^\Cf(X)$ and $\mu_0 = \diam^\Cf(X)$ in the previous lemma.

% Later we will use $\sep^\Cf(X)$ and $\diam^\Cf(X)$ but first we have to show an important property.

% \begin{lemma}{}{mutliplicativity_of_intrinsic_sizes}
% Let $\Cf: \M \to \C$ be surjective. Then for all $(X,d) \in \ob(\M)$ and $\lambda > 0$ we have
% \begin{align*}{}{}
%     \diam^\Cf(X,\lambda \cdot d) &= \lambda \cdot \diam^\Cf(X,\lambda) \\
%     \sep^\Cf(X,\lambda \cdot d) &= \lambda \cdot \sep^\Cf(X,\lambda)
% \end{align*}
% \end{lemma}

% \begin{proposition}{Classification of $\iso$ Clustering Functors}{}
% Let $\Cf: \iso \to \C$ be a clustering functor. Then there exists a choice ... moreover any such choice is a clustering functor.
% \end{proposition}
% In light of the previous proposition let now $\M \in \{\inj, \gen\}$


% A very important functor is the:
% \begin{definition}{Vietoris Rips Functor}{}
% $\Rf_\delta$
% \end{definition}

% \todo[to what degree should I discuss the relevance of VR in other domains?]



% \begin{proposition}{Vietoris Rips Functor}{}
% $\Rf_\delta$ is an excessive and surjective clustering functor.
% \end{proposition}

% \section{Representable Clustering Functors}


% $\Cf^\Omega$ is indeed a clustering functor. Let $f: X \to Y$ be a morphism and $x \sim y$. Then we have an $\omega \in \Omega$ and a $\phi \in \mor_\M(\omega, X)$ such that \eqref{eq:represented_clustering_functor_condition} holds. But then $f \circ \phi \in \mor_\M(\omega,Y)$ has the same properties and it follows that $f(x) \sim f(y)$.

% \begin{myremark}{}{}

% $$
% \Cf^\Omega(\omega) = (\omega, \{\omega\})
% $$

% We also have $\Cf^\Omega$ is the finest clustering functor such that $\omega \in \Omega$ get mapped into the trivial partition.
% \end{myremark}

% \begin{proposition}{Surjective Generative Model}{}
% \newresult
% Let $\Omega$ be a generative model then the following are equivalent:
% \begin{enumerate}
%     \item $\Cf^\Omega$ is surjective
%     \item $\exists \omega \in \Omega$ with $|\omega| = 2$ and for all $k \in \N$ we have 
%     $$
%     \sup\{\diam(\omega): \omega \in \Omega \text{ and } |\omega| = k\} < \infty.
%     $$ 
% \end{enumerate}
% furthermore such a generative model $\Omega$ is called \emph{surjective}.
% \end{proposition}





% \begin{theorem}{Excessive Clustering Schemes}{}
% Let $\M \in \{\inj, \gen\}$ then a clustering functor $\Cf: \M \to \C$ is representable if and only if it is excessive.
% \end{theorem}

% \begin{definition}{Generative Metric \cite{Carlsson2010}}{}
% Let $\Omega$ be a generative model. Let $(X,d)$ be a metric space, we define a symmetric positive definite function $W^\Omega: X \times X \to [0,+\infty]$. With:
% \begin{equation*}
% W^\Omega(x,y) := \begin{cases}
%     1/2 & 
%     \begin{array}{cc}
%          \text{if } \exists \, \omega \in \Omega \ \exists \, \phi \in \mor_{\M}(\omega, (X,d))\\
%          \text{ such that } \{x,y\} \subset \image(\phi)
%     \end{array}\\
%     1 & \hfil \text{else}
% \end{cases}
% \end{equation*}
% \end{definition}




% \begin{myremark}{}{}
% Let $d_{W^\Omega}$ be the shortest path distance metric \ref{def:shortest_dist_metric} then $(X,d) \mapsto (X,d_{W^\Omega})$ induces a functor $\Tf^\Omega: \M \to \M$.
% \end{myremark}

% \begin{theorem}{Factorizing Theorem}{}
% Let $\Omega$ \footnote{\cite{Carlsson2010} provides a construction of $\Tf^\Omega$ that works for finite generative models.} be a generative model then we have
% \begin{equation*}
% \Cf^\Omega = \Rf_1 \circ \mathfrak{T}^\Omega
% \end{equation*}
% \end{theorem}

% \section{Splitting Clustering Functors}

% \begin{definition}{Splitting}{}
% We say that a clustering functor $\Cf: \M \to \C$ is \emph{splitting} at $\delta_0 > 0$ if we have

% \begin{enumerate}
%     \item $\Cf(\Delta_2(\delta))$ is trivial for all $\delta < \delta_0$.
%     \item $\Cf(\Delta_2(\delta))$ is discrete for all $\delta \geq \delta_0$.
% \end{enumerate}

% \todo[check inequalities, check with def of vietoris-rips]
% \end{definition}

% \begin{proposition}{}{splitting_inj}
% Let $\Cf: \inj \to \C$ be a clustering functor that splits at $\delta > 0$. Then we have $\Rf_\delta \refines \Cf$.
% \end{proposition}

% \begin{proposition}{}{}
% Let $\Cf: \gen \to \C$ be a clustering functor then $\Cf$ is splitting at $\delta > 0$ if and only if $\Rf_\delta = \Cf$.
% \end{proposition}

% \begin{proof}
% To show $\Rf_\delta \refines \Cf$ we can adapt the proof of the previous proposition \ref{prop:splitting_inj}. It therefore remains to show $\Cf \refines \Rf_\delta$. \todo
% \end{proof}

% \begin{lemma}{}{}
% \newresult
% Let $\Cf: \M \to \C$ be a surjective clustering functor. Let $(X,d) \in \ob(\M)$ then there exists $\lambda \geq \lambda' > 0$ such that
% \begin{enumerate}
%     \item $\Cf(X, \lambda \cdot d)$ is discrete
%     \item $\Cf(X, \lambda' \cdot d)$ is trivial
% \end{enumerate}
% \end{lemma}

% \begin{corollary}{}{}
% \newresult

% \todo[maybe we can remove this corollary and inline it into the proof of the next]
% Let $\Cf: \M \to \C$ be a surjective clustering functor. Then there exists a $\delta > 0$ such that $\Cf$ splits at $\delta$.
% \end{corollary}

% \begin{corollary}{}{}
% \newresult
% Let $\M \in \{\gen, \inj\}$ and $\Cf: \M \to \C$ a surjective clustering functor then $\Cf \refines \Rf_\delta$ for some $\delta > 0$. More over we have $\Cf = \Rf_\delta$ if $\M = \gen$
% \end{corollary}

% \section{Scaling Invariant Clustering Functors}

% \begin{definition}{Scale Invariance}{scale_invariance}
% A clustering functor $\Cf: \M \to \C$ is called \emph{scale invariant} if for all $\lambda > 0$ and $(X,d) \in \ob(\M)$ we have:
% \begin{equation*}
%     \Cf((X,d)) = \Cf(X,\lambda \cdot d)).
% \end{equation*}
% \end{definition}

% \begin{proposition}{Scale Invariance in $\gen$}{}
% Let $\Cf: \gen \to \C$ be a scale invariant functor then either
% \begin{enumerate}
%     \item $\Cf(X,d)$ is trivial for all $(X,d) \in \ob(\gen)$
%     \item $\Cf(X,d)$ is discrete for all $(X,d) \in \ob(\gen)$
% \end{enumerate}
% \end{proposition}

% \begin{proof}
% $\Delta_2(\delta) \to X \to \Delta_2(\delta')$
% \end{proof}

% \begin{proposition}{Scale Invariance in $\inj$}{}
% Let $\Cf: \inj \to \C$ be a scale invariant functor then...
% \end{proposition}

% \begin{proof}
% $\Delta_{|X|}(\delta) \to X \to \Delta_{|X|}(\delta')$
% \end{proof}
