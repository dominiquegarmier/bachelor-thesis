\chapter{Preliminaries and Notations}

\section{Metric Spaces}
From now on, unless otherwise mentioned, \todo[is it ever?] metric spaces $(X,d)$ are finite \ie\ $|X| < \infty$.

\begin{notation}{Metric Spaces}{}
When there is no risk of confusion we will simply write $X$ denoting a metric space as opposed to $(X,d)$
\end{notation}

\begin{definition}{Separation and Diameter}{}
The \emph{separation} of a metric space $(X,d)$ is given by
\begin{equation*}
\sep(X,d) = \sep(X) := \inf\{d(x,y): x,y \in X, x \neq y\} \in \R_{\geq 0} \sqcup \{\infty\}.
\end{equation*}
We use the convention that $\inf \emptyset = \infty$.
\end{definition}

\begin{definition}{Symmetric Positive Definitie Function}{}
$$
W: X \times X \to [0, +\infty]
$$
is called \emph{connected} if ...
\end{definition}

\begin{definition}{Shortest Distance Metric}{shortest_dist_metric}
\end{definition}

\begin{definition}{Diameter}{}
The \emph{diameter} of a metric space $(X,d)$ is given by
\begin{equation*}
\diam(X,d) = \diam(X) := \sup\{d(x,y): x,y \in X\}
\end{equation*}
\end{definition}

\begin{myremark}{}{}
Since in our setting $X$ is finite we have $\sep(X) > 0$. And both the infimum and supremum in $\sep(X)$ and $\diam(X)$ are realized, provided $\sep(X) \neq \infty$.
\end{myremark}

\begin{defprop}{Minimal Sub-Dominant Ultrametric}{}
\end{defprop}

\section{Notation}
\begin{tabular}{c c}
    $\mathcal{C,H}$ & category, ... \\
    $\mathfrak{F,C,R}$ & functor, ... \\ 
\end{tabular}

