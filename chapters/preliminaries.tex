\chapter{Preliminaries and Notations}


\todo[better formulation]
This chapter serves as a concise place to find definitions which will be used throughout. Feel free to skip it and refer back to it when needed.

\section{Equivalence Relations}

\begin{definition}{}{}
    \todo[should i keep this?]

    Let $X$ be a set, a \emph{relation} on $X$ is a subset $R \subseteq X \times X$. And we write $x \sim_R y$ if $(x,y) \in R$.
    A relation with the following properties is called an \emph{equivalence relation}
    \begin{enumerate}
        \item \emph{Reflexivity}: $\forall x \in X: x \sim_R x$
        \item \emph{Symmetry}: $\forall x,y \in X: x \sim_R y \implies y \sim_R x$
        \item \emph{Transitivity}: $\forall x,y,z \in X: x \sim_R y \text{ and } y \sim_R z \implies x \sim_R z$
    \end{enumerate}
    We will often omit the explicit reference to the set $R$ and simply write $\sim$ or use other symbols.
\end{definition}

\begin{definition}{}{}
    Given a relation $R$ on a set $X$ the \emph{transitive closure} of $X$ is the minimal (with respect to $\subseteq$) $R^+ \subset X \times X$ such that $R \subset R^+$ and $R^+$ is transitive.
\end{definition}

In particular if $R$ is reflexive and symmetric then $R^+$ is an equivalence relation. We will use this fact as often time we will simply define a symmetric and reflexive relation then taking its transitive closure.

\todo[characterization with paths]

\section{Metric Spaces}
As a quick reminder:
\begin{definition}{Metric Space}{}
A metric space $(X,d)$ is a set $X$ and a metric \Ie\ a map
\begin{equation*}
d: X \times X \to [0, +\infty],
\end{equation*}
such that for all $x,y,z \in X$ the following hold:
\begin{enumerate}
    \item $d(x,y) = 0$ if and only if $x = y$
    \item $d(x,y) = d(y,x)$
    \item $d(x,z) \leq d(x,y) + d(y,z)$
\end{enumerate}
\end{definition}
From now on we primarily work with finite non-empty metric spaces, that is $0 < |X| < \infty$.
Furthermore, when there is no risk of confusion we will simply write $X$ denoting a metric space as opposed to $(X,d)$.

\begin{definition}{}{}
The \emph{separation} of a metric space $(X,d)$ is given by
\begin{equation*}
\sep(X,d) := \sep(X) := \inf\{d(x,y): x,y \in X, x \neq y\}
\end{equation*}
and the \emph{diameter}
\begin{equation*}
\diam(X,d) := \diam(X) := \sup\{d(x,y): x,y \in X\}
\end{equation*}
We use the convention that $\inf \emptyset = \infty$.
\end{definition}

If $X$ is finite non-empty we have $\diam(X) \geq \sep(X) > 0$ provided.

\subsection{Constructing Metrics}
Later on we will have to construct metrics with very particular properties. The easiest way to do this is to start with symmetric positive (definite) functions. We present two ways such a function can induce a metric.

\begin{definition}{}{shortest_dist_metric}
We say that a symmetric positive definite function
$$
W: X \times X \to [0, +\infty]
$$
\emph{represents a connected graph} or is \emph{connected} if for any $x,y \in X$ there exists a sequence $x = x_0, \dots, x_n = y$ such that $W(x_{i-1}, x_{i}) < \infty$ for all $i = 1, \dots, n$.
\end{definition}

\begin{definition}{}{}
Given a symmetric positive definite function $W$ representing a connected graph we can define the shortest distance metric
\begin{equation*}
    d^W(x,y) := \inf\left\{ \sum_{k = 1}^N W(x_{k-1}, x_k): N \in \N \text{ and } x = x_0, \dots, x_N = y\right\}
\end{equation*}
\end{definition}



\begin{defprop}{}{}
\source[we only present this here since its in the paper]

Minimal Sub-Dominant Ultrametric
\end{defprop}


\begin{definition}{Simplex $\Delta_k(\delta)$}{}
    \todo[move this to a better place]
\end{definition}