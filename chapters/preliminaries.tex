\chapter{Preliminaries and Notations}
\label{chapter__preliminaries}

This chapter serves as a concise place to find definitions which will be used throughout.
\section{Equivalence Relations}

\begin{definition}{}{}
    Let $X$ be a set, a \emph{relation} on $X$ is a subset $R \subseteq X \times X$. And we write $x \sim_R y$ if $(x,y) \in R$.
    A relation with the following properties is called an \emph{equivalence relation}:
    \begin{enumerate}
        \item \emph{Reflexivity}: $\forall x \in X: x \sim_R x$
        \item \emph{Symmetry}: $\forall x,y \in X: x \sim_R y \implies y \sim_R x$
        \item \emph{Transitivity}: $\forall x,y,z \in X: x \sim_R y \text{ and } y \sim_R z \implies x \sim_R z$
    \end{enumerate}
    We will often omit the explicit reference to the set $R$ and simply write $\sim$ or use other symbols.
\end{definition}

When describing an equivalence relation it is often easier to describe a minimal set of conditions the equivalence relation must satisfy. More formally, we talk about taking the \emph{transitive closure} of a relation.

\begin{definition}{}{}
    Given a relation $R$ on a set $X$ the \emph{transitive closure} of $R$ is the minimal (with respect to $\subseteq$) $R^+ \subset X \times X$ such that $R \subset R^+$ and $R^+$ is transitive.
\end{definition}

In particular, if $R$ is reflexive and symmetric then $R^+$ is an equivalence relation.
We will use this fact and simply define a symmetric and reflexive relation then taking its transitive closure.
More precisely, we have that $x \sim_{R^+} y$ if there exists a sequence $x = x_0, x_1, \dots, x_n = y$ such that $x_i \sim_R x_{i+1}$ for all $i = 0, \dots, n-1$ \cite[p.~337]{Lidl1997-kc}.

\section{Metric Spaces}
\begin{definition}{}{}
A \emph{metric space} $(X,d)$ is a set $X$ and a metric \ie\ a map
\begin{equation*}
d: X \times X \to [0, +\infty],
\end{equation*}
such that for all $x,y,z \in X$ the following hold:
\begin{enumerate}
    \item $d(x,y) = 0$ if and only if $x = y$
    \item $d(x,y) = d(y,x)$
    \item $d(x,z) \leq d(x,y) + d(y,z)$
\end{enumerate}
\end{definition}
Furthermore, when there is no risk of confusion we will simply write $X$ denoting a metric space as opposed to $(X,d)$.

\begin{definition}{}{}
The \emph{separation} of a metric space $(X,d)$ is given by
\begin{equation*}
\sep(X,d) := \sep(X) := \begin{cases}
    0 & \text{if } |X| \le 1\\
    \inf\{d(x,y): x,y \in X, x \neq y\} & \text{otherwise}
\end{cases}
\end{equation*}
and the \emph{diameter}
\begin{equation*}
\diam(X,d) := \diam(X) := \sup\{d(x,y): x,y \in X\}.
\end{equation*}
\end{definition}

Notice that for any metric space $(X,d)$ we have
$$
\diam(X) \geq \sep(X) \ge 0.
$$
From now on any metric space is assumed to be finite non-empty \ie\ $0 < |X| < \infty$.


\begin{definition}{}{}
    Let $k \in \N$ and $\delta > 0$. Then we define the \emph{$k$-simplex} with size $\delta$ as the metric space $\Delta_k(\delta) := (X,d)$ where $X = \{1, \dots, k\}$ and the metric is given by
    $$
    d(i,j) := \begin{cases}
        0 & \text{if } i = j\\
        \delta & \text{otherwise}
    \end{cases} \quad \forall i,j \in X.
    $$
\end{definition}