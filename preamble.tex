%% input encoding (utf8 allows Umlauts and accents directly)
\usepackage[T1]{fontenc}
\usepackage[utf8]{inputenc}
\usepackage{lmodern}

%% language
\usepackage[english]{babel}
\selectlanguage{english}

%% figures
\usepackage{xcolor}
\usepackage{graphicx}
\usepackage{tikz}
\usepackage{pgfplots}
\usepackage{tikz-cd}

\usetikzlibrary{arrows}
\usetikzlibrary{arrows.meta}
\usetikzlibrary{decorations.markings}
\usetikzlibrary{decorations.pathmorphing}
\usetikzlibrary{calc}
\usetikzlibrary{intersections}
\usetikzlibrary{shapes.geometric}
\usetikzlibrary{quotes}

\tikzset{>={Latex[length=6pt,width=5pt]}}

\pgfplotsset{compat=1.16}

\definecolor{darkgreen}{rgb}{0,.6,0.2}

%% color scheme
%% https://coolors.co/00ad6b-2282e2-8400b8-dc2c3d-e6a100-8f6b28
\definecolor{myblue}{RGB}{34, 130, 226}
\definecolor{myred}{RGB}{220, 44, 61}
\definecolor{myyellow}{RGB}{230, 161, 0}
\definecolor{mygreen}{RGB}{0, 173, 107}
\definecolor{mygray}{RGB}{184, 184, 184}
\definecolor{mypurple}{RGB}{71, 0, 99}
\definecolor{mybrown}{RGB}{143,107,40}

%% math
\usepackage{amsmath}
\usepackage{amsthm}
\usepackage{amssymb}
\usepackage{mathrsfs}
\usepackage{mathtools}
\usepackage{tensor}
\usepackage{mathrsfs}
\usepackage{array}
\usepackage{extarrows}
\usepackage{bigints}
\usepackage{bm}
\usepackage{nicematrix}

\numberwithin{equation}{section}

%% formatting
\usepackage{titlesec}
\usepackage{enumitem}
\usepackage{needspace}
\usepackage{caption}

\setlength{\parindent}{0em}
\setlength{\parskip}{0.5em}

\captionsetup{width=0.9\textwidth}

\renewcommand*{\arraystretch}{1.2}

\usepackage{hyperref}
\hypersetup{
	colorlinks,
	linkcolor={myblue},
	citecolor={myred},
	urlcolor={mygreen}
}


%% theorems etc.
\usepackage{tcolorbox}
\tcbuselibrary{theorems}
\tcbuselibrary{skins}
\tcbuselibrary{breakable}

\newtcbox{\inlinebox}[1][mygray]{on line,
arc=3pt,colback=#1!3,colframe=#1,
before upper={\rule[-3pt]{0pt}{10pt}},boxrule=1pt,
boxsep=0pt,left=2pt,right=2pt,top=1pt,bottom=.5pt}

\newcommand{\exe}{\inlinebox[myred]{exercise}}
\newcommand{\exes}{\inlinebox[myred]{exercises}}

\newcommand*{\todo}[1][]{\inlinebox[mypurple]{\textbf{TODO} #1}}
\newcommand*{\source}[1][]{\inlinebox[myblue]{\textbf{SOURCE} #1}}

\newtcbtheorem[number within=chapter]{definition}{Definition}
{
	enhanced,breakable,
	frame hidden,
	colframe=myred,
	colback=myred!3,
	coltitle=black,
	fonttitle=\bfseries,
	colbacktitle=myred!3,
	borderline={0.2mm}{0mm}{myred},
	attach boxed title to top text left,
	boxed title style={boxrule=0.2mm,skin=enhancedfirst jigsaw,bottom=0mm}
}
{def}

\newtcbtheorem[use counter from=definition]{defprop}{Definition \& Proposition}
{
	enhanced,breakable,
	frame hidden,
	colframe=myred,
	colback=myred!3,
	coltitle=black,
	fonttitle=\bfseries,
	colbacktitle=myred!3,
	borderline={0.2mm}{0mm}{myred},
	attach boxed title to top text left,
	boxed title style={boxrule=0.2mm,skin=enhancedfirst jigsaw,bottom=0mm}
}
{defprop}

\newtcbtheorem[use counter from=definition]{notation}{Notation}
{
	enhanced,breakable,
	frame hidden,
	colframe=myred,
	colback=myred!3,
	coltitle=black,
	fonttitle=\bfseries,
	colbacktitle=myred!3,
	borderline={0.2mm}{0mm}{myred},
	attach boxed title to top text left,
	boxed title style={boxrule=0.2mm,skin=enhancedfirst jigsaw,bottom=0mm}
}
{not}

\newtcbtheorem[use counter from=definition]{example}{Example}
{
	enhanced,breakable,
	frame hidden,
	colframe=myyellow,
	colback=myyellow!3,
	coltitle=black,
	fonttitle=\bfseries,
	colbacktitle=myyellow!3,
	borderline={0.2mm}{0mm}{myyellow},
	attach boxed title to top text left,
	boxed title style={boxrule=0.2mm,skin=enhancedfirst jigsaw}
}
{exa}


\newtcbtheorem[use counter from=definition]{myremark}{Remark}
{
	enhanced,breakable,
	frame hidden,
	colframe=myyellow,
	colback=myyellow!3,
	coltitle=black,
	fonttitle=\bfseries,
	colbacktitle=myyellow!3,
	borderline={0.2mm}{0mm}{myyellow},
	attach boxed title to top text left,
	boxed title style={boxrule=0.2mm,skin=enhancedfirst jigsaw}
}
{rem}

\newtcbtheorem[use counter from=definition]{proposition}{Proposition}
{
	enhanced,breakable,
	frame hidden,
	colframe=myblue,
	colback=myblue!3,
	coltitle=black,
	fonttitle=\bfseries,
	colbacktitle=myblue!3,
	borderline={0.2mm}{0mm}{myblue},
	attach boxed title to top text left,
	boxed title style={boxrule=0.2mm,skin=enhancedfirst jigsaw}
}
{prop}

\newtcbtheorem[use counter from=definition]{fact}{Fact}
{
	enhanced,breakable,
	frame hidden,
	colframe=mygreen,
	colback=mygreen!3,
	coltitle=black,
	fonttitle=\bfseries,
	colbacktitle=mygreen!3,
	borderline={0.2mm}{0mm}{mygreen},
	attach boxed title to top text left,
	boxed title style={boxrule=0.2mm,skin=enhancedfirst jigsaw}
}
{fact}

\newtcbtheorem[use counter from=definition]{lemma}{Lemma}
{
	enhanced,breakable,
	frame hidden,
	colframe=mygreen,
	colback=mygreen!3,
	coltitle=black,
	fonttitle=\bfseries,
	colbacktitle=mygreen!3,
	borderline={0.2mm}{0mm}{mygreen},
	attach boxed title to top text left,
	boxed title style={boxrule=0.2mm,skin=enhancedfirst jigsaw}
}
{lem}

\newtcbtheorem[use counter from=definition]{theorem}{Theorem}
{
	enhanced,breakable,
	frame hidden,
	colframe=mypurple,
	colback=mypurple!3,
	coltitle=black,
	fonttitle=\bfseries,
	colbacktitle=mypurple!3,
	borderline={0.2mm}{0mm}{mypurple},
	attach boxed title to top text left,
	boxed title style={boxrule=0.2mm,skin=enhancedfirst jigsaw}
}
{thm}

\newtcbtheorem[use counter from=definition]{corollary}{Corollary}
{
	enhanced,breakable,
	frame hidden,
	colframe=mybrown,
	colback=mybrown!3,
	coltitle=black,
	fonttitle=\bfseries,
	colbacktitle=mybrown!3,
	borderline={0.2mm}{0mm}{mybrown},
	attach boxed title to top text left,
	boxed title style={boxrule=0.2mm,skin=enhancedfirst jigsaw}
}
{cor}

\newtcbtheorem[use counter from=definition]{exercise}{Exercise}
{
	enhanced,breakable,
	frame hidden,
	colframe=myyellow,
	colback=myyellow!3,
	coltitle=black,
	fonttitle=\bfseries,
	colbacktitle=myyellow!3,
	borderline={0.2mm}{0mm}{myyellow},
	attach boxed title to top text left,
	boxed title style={boxrule=0.2mm,skin=enhancedfirst jigsaw}
}
{exe}

\def\R{\mathbb R}
\def\Cb{\mathbb C}
\def\N{\mathbb N}
\def\Z{\mathbb Z}
\def\Q{\mathbb Q}
\def\E{\mathbb E}

\def\mor{\mathrm {Mor}}
\def\ob{\mathrm {Ob}}

% partitions
\def\P{\mathfrak{P}}

% categories
\def\M{\mathcal{M}}
\def\C{\mathcal{C}}
\def\H{\mathcal{H}}
\def\A{\mathcal{A}}
\def\B{\mathcal{B}}

\def\gen{{\mathcal{M}_\mathrm{gen}}}
\def\inj{{\mathcal{M}_\mathrm{inj}}}
\def\iso{{\mathcal{M}_\mathrm{iso}}}

% functors
\def\Cf{\mathfrak{C}}
\def\Rf{\mathfrak{R}}
\def\Jf{\mathfrak{J}}
\def\Ff{\mathfrak{F}}
\def\Hf{\mathfrak{H}}
\def\Tf{\mathfrak{T}}

\def\id{\mathrm{id}}

\def\refines{\preceq_\mathrm{ref}}

% latin stuff
\def\eg{\emph{e.g.}}
\def\Eg{\emph{E.g.}}
\def\ie{\emph{i.e.}}
\def\Ie{\emph{I.e.}}
\def\etc{\emph{etc.}}
\def\st{\text{s.t.}}

\def\apriori{\emph{a priori}}
\def\Apriori{\emph{A priori}}

\def\sep{\mathrm{sep}}
\def\diam{\mathrm{diam}}

\DeclareMathOperator*{\argmin}{argmin}
\DeclareMathOperator*{\argmax}{argmax}
\DeclareMathOperator*{\image}{im}
\DeclareMathOperator*{\kernel}{ker}